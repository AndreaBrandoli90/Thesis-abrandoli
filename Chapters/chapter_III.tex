\chapter{Nakamoto Consensus in Bitcoin}
\label{chpr:btc}
Bitcoin is a distributed, decentralized, peer-to-peer electronic payment system based on cryptographic proof instead of trust, allowing transactions between two counterparts without the need for a trusted third party. However, in late 2008, when the white paper was published by his author Satoshi Nakamoto (metti nota sul nome), it lacked a formalization of the protocol and of the guarantees it claimed to provide.

\bigskip
\noindent
This chapter delves into the core innovation behind Bitcoin, i.e. \textit{Nakamoto consensus}, term that is commonly used to refer to Bitcoin's novel consensus mechanism, which allows mutually distrusting pseudonymous identities to reach eventual agreement.

\bigskip
\section{Bitcoin \& Eventual Consistency}
By its very nature, the Bitcoin network is subject to a type of failure called \textit{network partition}, where a network splits into at least two parts that cannot communicate with each other, often due to software bugs, incompatible protocol versions, or simply network disconnections.
Thus, Bitcoin is inherently characterized by a trade-off between \textit{consistency}, \textit{availability} and \textit{partition tolerance}. Let us be more precise.
\begin{mydef} {\bf (consistency)}.
    All nodes in the system agree on the current state of the system.
\end{mydef}
\begin{mydef} {\bf (availability)}.
    The system is operational and instantly processing incoming requests.
\end{mydef}
\begin{mydef} {\bf (partition tolerance)}.
    Partition tolerance is the ability of a distributed system to continue operating correctly even in the presence of a network partition.
\end{mydef}

\bigskip
\noindent
In practice, only two of these properties can be reached simultaneously; a theorem by Brewer proves this result.
\begin{thm} {\bf (CAP theorem)}.
    It is impossible for a distributed system to simultaneously provide consistency, availability and partition tolerance. A distributed system can satisfy any two of these but not all three.
\end{thm}
\begin{proof}
    Let us assume two nodes that share some state. The nodes belongs to different partitions, so they cannot communicate each other. Assume a request wants to update the state and contacts one of the two nodes. The node may either: 1) update its local state, resulting in a situation of inconsistent states, or 2) not update, resulting in a system no longer available for updates.
\end{proof}

\bigskip
\noindent
Recalling the Definition \ref{def:consensus} of consensus and the properties it must satisfy, a clever intuition of Nakamoto is to weaken the agreement property to hold probabilistically and not deterministically, in order to deal with an asynchronous network. In particular the aforementioned trade-off is in advantage of partition tolerance rather than consistency. In fact, state changes of the underlying transaction ledger (i.e. the blockchain), are rendered probabilistic and the decision on a specific value of the state reaches $Pr(1)$ when $\lim_{r \to \infty}$, where $r$ is the number of rounds in the consensus protocol.

\bigskip
\noindent
Therefore, we can define Bitcoin as an example of \textit{eventual consistency}.
\begin{mydef}{\bf (eventual consistency)}
    If no new updates to the shared state are issued, then eventually the system is in a quiescent state, i.e., no more messages need to be exchanged between nodes, and the shared state is consistent.
\end{mydef}

\bigskip
\noindent
Questo è un primo passo per comprendere come nakamoto risolve il consenso distribuito in Bitcoin, per comprenderlo a pieno serve innanzitutto un background crittografico e lo studio di possibili attacchi al network dall'esterno per domostrare la solidità e l'immutabilità della blockchain sottostante (hash pointers)...quindi hash, commit, merkle trees, miner incentives, pow.
