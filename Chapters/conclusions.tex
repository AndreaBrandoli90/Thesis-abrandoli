\chapter{Conclusions and Future Work}
\label{chpr:conclusions}
The aim of this thesis has been to give an explanation on how to strengthen the notarization process by means of a decentralized solution built upon a blockchain technology, in particular the Bitcoin blockchain, being the most secure and reliable decentralized system.

\bigskip
\noindent
To achieve this goal we first presented distributed system in general, emphasizing the advantages in using them and focusing on the consensus problem. Then we studied the consensus problem in presence of byzantine nodes in the case of an asynchronous network, putting ourselves in the same basic structure of Bitcoin. After presenting a famous result of Fisher, Lynch and Paterson \cite{Fischer:1985:IDC:3149.214121}, which proves the impossibility of distributed consensus with one faulty process, we saw how brilliantly Satoshi Nakamoto solves the consensus problem in Bitcoin, combining proof-of-work and economic incentives, thus making the underlying blockchain almost immutable. Hence, convinced of the security and immutability of the Bitcoin blockchain, we started presenting one of the most important non monetary applications of such technology, that is timestamping of digital documents. First we learned how the timestamping procedure on blockchain works, underlining the cryptographic operations needed. Then we introduced \textit{OpenTimestamps}, an open standard consisting in a set of operations for creating provable blockchain timestamps that can be independently verified. It also solves a scalability problem encountered in the first notarization attempts on blockchain, aggregating digests to be timestamped in a merkle tree. We also gave an overview on the usage of the \textit{OpenTimestamps} Python client and the web interface, enticing the reader to try timestamping a document. Finally we 