\chapter{Bitcoin from the command line}
\label{app:A}
This appendix is aimed at providing a technical walkthrough the Bitcoin network from the command line point of view, supposing a Unix-like operating system. Initially, we will configure a machine to launch a Bitcoin node; then we will introduce some of the most important RPC calls\footnote{You can find the original Bitcoin client/API calls list here: \url{https://en.bitcoin.it/wiki/Original_Bitcoin_client/API_calls_list}.} in order to perform basic operations.

\bigskip
\section{Running a Bitcoin Core node}
To get started with Bitcoin, you first need to join the network running a node. Following these simple steps\footnote{The presented guide is a simplified version of a very detailed open-source guide by C. Allen \cite{LearningBitcoinFromTheCommandLine}, conceived for a local installation of Bitcoin Core.} will make the first experience a lot easier.
\begin{itemize}
\item Setup variables (for an easy installation):
\bigskip
\begin{lstlisting}
 $ export BITCOIN=bitcoin-core-0.17.1
\end{lstlisting}
\item Download relevant files (Remark: every time you find username replace it with your personal one):
\bigskip
\begin{lstlisting}
 $ wget https://bitcoin.org/bin/$BITCOIN/$BITCOINPLAIN-x86_64-linux-gnu.tar.gz -O ~username/$BITCOINPLAIN-x86_64-linux-gnu.tar.gz
 $ wget https://bitcoin.org/bin/$BITCOIN/SHA256SUMS.asc -O ~username/SHA256SUMS.asc
 $ wget https://bitcoin.org/laanwj-releases.asc -O ~username/laanwj-releases.asc
\end{lstlisting}
\item Verify Bitcoin signature (in order to verify that your Bitcoin setup is authentic):
\bigskip
\begin{lstlisting}
 $ /usr/bin/gpg --import ~username/laanwj-releases.asc
 $ /usr/bin/gpg --verify ~username/SHA256SUMS.asc
\end{lstlisting}
Amongst the info you get back from the last command should be a line telling \enquote{Good signature}, do not care the rest.
\item Verify Bitcoin SHA (Secure Hash Algorithm) of the .tar file against the expected one:
\bigskip
\begin{lstlisting}
 $ /usr/bin/sha256sum ~username/$BITCOINPLAIN-x86_64-linux-gnu.tar.gz | awk '{print $1}'
 $ cat ~username/SHA256SUMS.asc | grep $BITCOINPLAIN-x86_64-linux-gnu.tar.gz | awk '{print $1}'
\end{lstlisting}
From this verification you must obtain the same number.
\item Install Bitcoin:
\bigskip
\begin{lstlisting}
 $ /bin/tar xzf ~username/$BITCOINPLAIN-x86_64-linux-gnu.tar.gz -C ~username
 $ sudo /usr/bin/install -m 0755 -o root -g root -t /usr/local/bin ~username/$BITCOINPLAIN/bin/*
 $ /bin/rm -rf ~username/$BITCOINPLAIN/
\end{lstlisting}
\item Create the configuration file:

\bigskip
\noindent
First, create the directory you want to put your data into: \bigskip
\begin{lstlisting}
 $ /bin/mkdir ~username/.bitcoin
\end{lstlisting}
Then create the configuration file (as you need it):
\bigskip
\begin{lstlisting}
 $ cat >> ~username/.bitcoin/bitcoin.conf << EOF
 # Accept command line and JSON-RPC commands
 server=1
 # Username for JSON-RPC connections
 rpcuser=invent_a_username_here
 # Password for JSON-RPC connections
 rpcpassword=invent_a_long_pwd_here
 # Enable Testnet chain mode
 testnet=1
 EOF
\end{lstlisting}
The \textit{bitcoin.conf} file is the default file which Bitcoin daemon reads when launched, and it contains instructions to configure the node (e.g. mainnet, testnet or regtest mode). In this example we define a \textit{rpcuser} and \textit{rpcpassword} to enable RPC, and we tell the node to synchronize with the Bitcoin Testnet chain.

\bigskip
\noindent
Finally, limit the permissions to your configuration file (not really needed, but more secure): \bigskip
\begin{lstlisting}
 $ /bin/chmod 600 ~username/.bitcoin/bitcoin.conf
\end{lstlisting}
\item Start the daemon\footnote{In multitasking computer operating systems, a \textit{daemon} is a computer program that runs as a background process, rather than being under direct control of an interactive user. Traditionally, the process names of a daemon end with the letter \textit{d}; \colorbox{Grey!10}{bitcoind} is the one which implements the Bitcoin protocol for remote procedure call (RPC) use.}:
\bigskip
\begin{lstlisting}
 $ bitcoind -daemon
\end{lstlisting}
\end{itemize}
We are now ready to be part of the Bitcoin network ! \\ 
In the next section we will learn how to manage the Bitcoin command line interface.

\bigskip

\section{Bitcoin-cli: command line interface}
In this section we will answer the following question: \enquote{\textit{What is RPC and which calls are the most used in Bitcoin?}}.

\bigskip
\noindent
Without further going into details regarding computer science, a RPC\footnote{For more details, see \url{https://en.wikipedia.org/wiki/Remote\_procedure\_call}.} (Remote Procedure Call) is a powerful technique for constructing distributed, client-server based applications. It is based on the extension of the conventional procedure calling, so that the called procedure need not exist in the same address space as the calling procedure. The two processes may be on the same system, or they may be on different systems with a network connecting them. Our interest is for the \colorbox{Grey!10}{bitcoin-cli}, a handy interface that lets the user send commands to the \colorbox{Grey!10}{bitcoind}. More specifically, it lets you send RPC commands to the \colorbox{Grey!10}{bitcoind}. Generally, the \colorbox{Grey!10}{bitcoin-cli} interface is much more clean and user friendly than trying to send RPC commands by hand, using \colorbox{Grey!10}{curl} or some other method.

\bigskip
\noindent
The first thing to do for a beginner user is to look for the full list of calls invoking the \colorbox{Grey!10}{help} command:
\bigskip
\begin{lstlisting}
 $ bitcoin-cli help
 
 == Blockchain ==
 getbestblockhash
 getblock "blockhash" ( verbosity ) 
 getblockchaininfo
 getblockcount
 getblockhash height
 getblockheader "hash" ( verbose )
 getblockstats hash_or_height ( stats )
 getchaintips
 getchaintxstats ( nblocks blockhash )
 getdifficulty
 getmempoolancestors txid (verbose)
 getmempooldescendants txid (verbose)
 getmempoolentry txid
 getmempoolinfo
 getrawmempool ( verbose )
 gettxout "txid" n ( include_mempool )
 gettxoutproof ["txid",...] ( blockhash )
 gettxoutsetinfo
 preciousblock "blockhash"
 pruneblockchain
 savemempool
 scantxoutset <action> ( <scanobjects> )
 verifychain ( checklevel nblocks )
 verifytxoutproof "proof"

 == Control ==
 getmemoryinfo ("mode")
 help ( "command" )
 logging ( <include> <exclude> )
 stop
 uptime

 == Generating ==
 generate nblocks ( maxtries )
 generatetoaddress nblocks address (maxtries)

 == Mining ==
 getblocktemplate ( TemplateRequest )
 getmininginfo
 getnetworkhashps ( nblocks height )
 prioritisetransaction <txid> <dummy value> <fee delta>
 submitblock "hexdata"  ( "dummy" )

 == Network ==
 addnode "node" "add|remove|onetry"
 clearbanned
 disconnectnode "[address]" [nodeid]
 getaddednodeinfo ( "node" )
 getconnectioncount
 getnettotals
 getnetworkinfo
 getpeerinfo
 listbanned
 ping
 setban "subnet" "add|remove" (bantime) (absolute)
 setnetworkactive true|false

 == Rawtransactions ==
 combinepsbt ["psbt",...]
 combinerawtransaction ["hexstring",...]
 converttopsbt "hexstring" ( permitsigdata iswitness )
 createpsbt [{"txid":"id","vout":n},...] [{"address":amount},{"data":"hex"},...] ( locktime ) ( replaceable )
 createrawtransaction [{"txid":"id","vout":n},...] [{"address":amount},{"data":"hex"},...] ( locktime ) ( replaceable )
 decodepsbt "psbt"
 decoderawtransaction "hexstring" ( iswitness )
 decodescript "hexstring"
 finalizepsbt "psbt" ( extract )
 fundrawtransaction "hexstring" ( options iswitness )
 getrawtransaction "txid" ( verbose "blockhash" )
 sendrawtransaction "hexstring" ( allowhighfees )
 signrawtransaction "hexstring" ( [{"txid":"id","vout":n,"scriptPubKey":"hex","redeemScript":"hex"},...] ["privatekey1",...] sighashtype )
 signrawtransactionwithkey "hexstring" ["privatekey1",...] ( [{"txid":"id","vout":n,"scriptPubKey":"hex","redeemScript":"hex"},...] sighashtype )
 testmempoolaccept ["rawtxs"] ( allowhighfees )

 == Util ==
 createmultisig nrequired ["key",...] ( "address_type" )
 estimatesmartfee conf_target ("estimate_mode")
 signmessagewithprivkey "privkey" "message"
 validateaddress "address"
 verifymessage "address" "signature" "message"

 == Wallet ==
 abandontransaction "txid"
 abortrescan
 addmultisigaddress nrequired ["key",...] ( "label" "address_type" )
 backupwallet "destination"
 bumpfee "txid" ( options ) 
 createwallet "wallet_name" ( disable_private_keys )
 dumpprivkey "address"
 dumpwallet "filename"
 encryptwallet "passphrase"
 getaccount (Deprecated, will be removed in V0.18. To use this command, start bitcoind with -deprecatedrpc=accounts)
 getaccountaddress (Deprecated, will be removed in V0.18. To use this command, start bitcoind with -deprecatedrpc=accounts)
 getaddressbyaccount (Deprecated, will be removed in V0.18. To use this command, start bitcoind with -deprecatedrpc=accounts)
 getaddressesbylabel "label"
 getaddressinfo "address"
 getbalance ( "(dummy)" minconf include_watchonly )
 getnewaddress ( "label" "address_type" )
 getrawchangeaddress ( "address_type" )
 getreceivedbyaccount (Deprecated, will be removed in V0.18. To use this command, start bitcoind with -deprecatedrpc=accounts)
 getreceivedbyaddress "address" ( minconf )
 gettransaction "txid" ( include_watchonly )
 getunconfirmedbalance
 getwalletinfo
 importaddress "address" ( "label" rescan p2sh )
 importmulti "requests" ( "options" )
 importprivkey "privkey" ( "label" ) ( rescan )
 importprunedfunds
 importpubkey "pubkey" ( "label" rescan )
 importwallet "filename"
 keypoolrefill ( newsize )
 listaccounts (Deprecated, will be removed in V0.18. To use this command, start bitcoind with -deprecatedrpc=accounts)
 listaddressgroupings
 listlabels ( "purpose" )
 listlockunspent
 listreceivedbyaccount (Deprecated, will be removed in V0.18. To use this command, start bitcoind with -deprecatedrpc=accounts)
 listreceivedbyaddress ( minconf include_empty include_watchonly address_filter )
 listsinceblock ( "blockhash" target_confirmations include_watchonly include_removed )
 listtransactions (dummy count skip include_watchonly)
 listunspent ( minconf maxconf  ["addresses",...] [include_unsafe] [query_options])
 listwallets
 loadwallet "filename"
 lockunspent unlock ([{"txid":"txid","vout":n},...])
 move (Deprecated, will be removed in V0.18. To use this command, start bitcoind with -deprecatedrpc=accounts)
 removeprunedfunds "txid"
 rescanblockchain ("start_height") ("stop_height")
 sendfrom (Deprecated, will be removed in V0.18. To use this command, start bitcoind with -deprecatedrpc=accounts)
 sendmany "" {"address":amount,...} ( minconf "comment" ["address",...] replaceable conf_target "estimate_mode")
 sendtoaddress "address" amount ( "comment" "comment_to" subtractfeefromamount replaceable conf_target "estimate_mode")
 setaccount (Deprecated, will be removed in V0.18. To use this command, start bitcoind with -deprecatedrpc=accounts)
 sethdseed ( "newkeypool" "seed" )
 settxfee amount
 signmessage "address" "message"
 signrawtransactionwithwallet "hexstring" ( [{"txid":"id","vout":n,"scriptPubKey":"hex","redeemScript":"hex"},...] sighashtype )
 unloadwallet ( "wallet_name" )
 walletcreatefundedpsbt [{"txid":"id","vout":n},...] [{"address":amount},{"data":"hex"},...] ( locktime ) ( replaceable ) ( options bip32derivs )
 walletlock
 walletpassphrase "passphrase" timeout
 walletpassphrasechange "oldpassphrase" "newpassphrase"
 walletprocesspsbt "psbt" ( sign "sighashtype" bip32derivs )

 == Zmq ==
 getzmqnotifications
\end{lstlisting}

\bigskip
\noindent
You can also type \colorbox{Grey!10}{bitcoin-cli help [command]} to be even more extensive about that command. For example:
\bigskip
\begin{lstlisting}
 $ bitcoin-cli help getmininginfo
 
 getmininginfo

 Returns a json object containing mining-related information.
 Result:
 {
   "blocks": nnn,             (numeric) The current block
   "currentblockweight": nnn, (numeric) The last block weight
   "currentblocktx": nnn,     (numeric) The last block transaction
   "difficulty": xxx.xxxxx    (numeric) The current difficulty
   "networkhashps": nnn,      (numeric) The network hashes per second
   "pooledtx": n              (numeric) The size of the mempool
   "chain": "xxxx",           (string) current network name as defined in BIP70 (main, test, regtest)
   "warnings": "..."          (string) any network and blockchain warnings
 }

 Examples:
 > bitcoin-cli getmininginfo 
 > curl --user myusername --data-binary '{"jsonrpc": "1.0", "id":"curltest", "method": "getmininginfo", "params": [] }' -H 'content-type: text/plain;' http://127.0.0.1:8332/
\end{lstlisting}
\bigskip
\noindent
Here is a list of the most general commands to get additional information about bitcoin data:
\bigskip
\begin{lstlisting}
 $ bitcoin-cli getblockchaininfo
 $ bitcoin-cli getmininginfo
 $ bitcoin-cli getnetworkinfo
 $ bitcoin-cli getnettotals
 $ bitcoin-cli getwalletinfo
\end{lstlisting}
\bigskip
\noindent
For example \colorbox{Grey!10}{bitcoin-cli getwalletinfo} gives the user precise information about the bitcoin wallet, like the confirmed and unconfirmed balance (nota a pie' pagina per spiegare cosa sia unconfirmed balance), and the total number of transactions in the wallet.
\bigskip
\begin{lstlisting}
 $ bitcoin-cli getwalletinfo
 {
  "walletname": "",
  "walletversion": 169900,
  "balance": 0.07825304,
  "unconfirmed_balance": 0.00000000,
  "immature_balance": 0.00000000,
  "txcount": 1,
  "keypoololdest": 1544714806,
  "keypoolsize": 999,
  "keypoolsize_hd_internal": 1000,
  "paytxfee": 0.00000000,
  "hdseedid": "972e6ac8d0104fec0d0e6c052d3876ada965c745",
  "hdmasterkeyid": "972e6ac8d0104fec0d0e6c052d3876ada965c745",
  "private_keys_enabled": true
}
\end{lstlisting}


