\chapter{Abstract}
Notarization of digital documents is traditionally achieved by a trusted certification authority, which represents a single point of failure. Thus, an application built upon a distributed system reaching consensus in a decentralized way, can greatly strengthen such process. In particular the Bitcoin blockchain, as the most reliable among the others, can be used as a notary to prove that a digital document existed prior to a certain time $t$ and in a certain state, a procedure called timestamping. \textit{OpenTimestamps} is an open-source project that aims to be a standard for timestamping. Additionally, it brilliantly solves a scalability issue found in some previous timestamping attempts on blockchain. The aim of this work is to give an explanation of the whole timestamping process, in particular by means of \textit{OpenTimestamps}. Finally, we developed a fully operating timestamping service, extending such protocol to achieve additional features.